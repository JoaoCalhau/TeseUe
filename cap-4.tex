%!TEX root = main.tex
\chapter{Our Approach}

\section{Introduction}

This section introduces our approach for modeling a Digital Forensics Problem as a Constraint Satisfaction Problem and reaching a valid solution using the Choco Solver. We describe how we modelled the problem as a CSP, the methodologies used to analyze and extract the information from the digital evidences, the data structures used to model the problem and how they are used to reach a solution.

\section{Methodology}

After acquiring the disk image to be analyzed, it is first processed with the help of tools from The Sleuth Kit~\cite{tsk}, first with Sorter and then with Mactime. Sorter creates multiple files with different names, each name being a pre-determined type of file, such as archive, executable or data. As for Mactime, it creates a single file containing time information about every file present in the file system. Examples of these files can be seen in Figure~\ref{fig:sorterOut} and Figure~\ref{fig:mactimeOut}. 

The files created by Sorter include all relevant information about the files present in the file system that is being analyzed, including: file path in the file system, the file type, the image name (from where the data was extracted) and the \INODE number, which is an internal representation of that particular file in the file system. As for the Mactime file, it only contains date information about each file. This information is parsed into a database that allows the data to be persistent. This database is described in detail in Section~\ref{database}. The data also passes through a caching system that tries to determine if the exact type of constraints have been applied to the file system being analyzed to find if it can skip the lengthily process of trying to find a solution. The caching system is described in detail in Section~\ref{caching}.

\begin{figure}
    \centering
    \includegraphics[width=120mm]{sorter_out.png}
    \caption{Example of Sorter output for executable files}
    \label{fig:sorterOut}
\end{figure}

\begin{figure}
    \centering
    \includegraphics[width=120mm]{mactimeOut.png}
    \caption{Example of Mactime output for a 4GB pen drive}
    \label{fig:mactimeOut}
\end{figure}

The whole program can be shortened into the small flow diagram seen in figure \ref{fig:diagram}.

\begin{figure}
    \centering
    \includegraphics[width=120mm]{diagram.png}
    \caption{Flow diagram}
    \label{fig:diagram}
\end{figure}

After pre-processing the disk image and persisting all necessary data, the digital forensics problem, which describes the items that being looked for, is modeled as a Constraint Satisfaction Problem (CSP). This CSP is then solved by a CSP solver, to reach a solution that satisfies the initial digital forensics problem, if it exists. A solution to such a CSP will be a set of file identifiers that match the digital forensics problem. It is necessary to use specific constraints to reach the solution mentioned before, which are used to restrict the domain of the variables. These constraints are described in Section~\ref{constraints}.

\section{Modeling}

The problem is modeled as a set of variables that represent the files that need to be found, according to the digital forensics problem. These files are represented as numerical values, which are the \INODE numbers extracted from the Sorter mentioned previously. Each of these variables are associated with a domain and a set of constraints which are applied to the variables. As previously stated these constraints were created specifically in the context of this work.

In this work we use the Choco Solver, which allows the use of four different types of variables to model the problems: Integers (IntVar), Booleans (BoolVar), Sets (SetVar) and Reals (RealVar). To model a digital forensics problem as a CSP, we decided to use  Set variables (SetVars).

In Choco Solver, SetVars are defined by a domain that is made of two separate domains, the Lower Bound (LB) and the Upper Bound (UB). The Lower Bound is a set of integers that must belong to every solution, while the Upper Bound is composed of the set of integers that may be part of the final solution \cite{SetVar}. In our case, when creating the variable with which we are going to work with, the Lower Bound will be left empty, because we do not know what files we want yet. As for the  Upper Bound, it will be composed of all the \INODES found during the parsing phase.

\floatname{algorithm}{Listing}

\begin{algorithm}
    \caption{Modelling of a Digital Forensics CSP}
    \label{modelling}
    \begin{algorithmic}
        \State{}
        \State{$CSP = (V, D, C)$}
        \State{}
        \State{$V =\{V_1,V_2,\ldots,V_n\}$}
        \State{}
        \State{$D = \{D_1,D_2,\ldots,D_n\},\quad \forall D_i \in D : D_i = \{Y_1,Y_2,\ldots,Y_z\}$}
        \State{}
        \State{$C=\{C_1(V_i,\ldots,V_j),\ldots,C_k(V_i,\ldots,V_j)\},$}
        \State{$\qquad \qquad \qquad \qquad \qquad \forall C_k \in C : C_k = \{CP_1(V_i,\ldots,V_j),\ldots,CP_z(V_i,\ldots,V_j)\}$}
        \State{}
    \end{algorithmic}
\end{algorithm}

Listing~\ref{modelling} presents a formal representation of a digital forensics CSP, represented by the triple $CSP = (V,D,C)$. $V$ is the set of variables, which represents the files to be found, $D$ is the set of domains for each variable which, where each $D_i = \{Y_1,Y_2,\ldots,Y_z\}$ is a set of integer values that represent the \INODES present in the file system; and $C$ is the set of constraints which restricts the domain of each variable, where each $C_k$ is mapped into a specific constraint propagator.

\section{Constraints}
\label{Constraints}

To reach a solution, Choco solver makes use of constraints which are implemented as propagators. A propagator declares a filtering algorithm that can be applied to the variables that models the problem, in order to reduce their domain~\cite{Propagator}. In the context of this work, we implemented the following propagators:

\begin{enumerate}
    \item \textbf{File type:} restricts our domain according to the given type of file passed as argument.
    \item \textbf{File path:} restricts our domain according to the given path passed as argument.
    \item \textbf{Word search:} restricts our domain according to the given word passed as argument.
    \item \textbf{NIST:} restricts our domain according to the NIST database.
    \item \textbf{Date:} restricts our domain according to the given date passed as argument.
\end{enumerate}

All these propagators were implemented to work with the variables used to model the problem: SetVars.

To create the propagators, we had to implement the following methods: \texttt{propagate} and \texttt{isEntailed}. The method \texttt{propagate} should restrict the domain according to what we need, while the method \texttt{isEntailed} informs the propagator if a problem has a solution or not, or if it is not possible to determine if a solution exists. These methods are described in detail in Listings \ref{propagate} and \ref{isEntailed}.

\floatname{algorithm}{Listing}

\begin{algorithm}
    \caption{propagate method}
    \label{propagate}
    \begin{algorithmic}
        \For{each value in UB}
            \If{value does not belong to desired structure}
                \State{Remove value from UB}
            \EndIf
        \EndFor
    \end{algorithmic}
\end{algorithm}

\begin{algorithm}
    \caption{isEntailed method}
    \label{isEntailed}
    \begin{algorithmic}
        \If{UB is empty}
            \State{Problem is impossible to solve}
        \Else
            \State{Problem has possible solution}
        \EndIf
    \end{algorithmic}
\end{algorithm}

\subsection{Type, Path and Date Propagators}

The Type propagator was the first propagator created and it takes the Upper Bound of the SetVar, iterates over it and removes any \INODE that does not belong to the type we are looking for. Path and Date propagators work in the same way. The main difference is what they are trying to restrict: the Type propagator receives the type of file we want to restrict and iterates over the Upper Bound to remove every \INODE that does not belong to that type of file, while the Path propagator receives a path and, similarly to the Type propagator, iterates over the Upper Bound and removes every \INODE that does not belong to that path.

\subsection{Word Search Propagator}

This propagator relies on the Unix4j~\cite{Unix4j} Java library. This library implements most of the Linux bash commands in native Java. We use this library since it provides efficient methods to find contents inside a file in any file system.

This propagator works in a similar way to the ones previously described. It iterates over the Upper Bound and removes any file that does not contain the word we passed as argument. Also, since this is the last constraint ever being run, we add the \INODES that passed through the propagator to the Lower Bound, because at this moment we're certain these will belong to the final solution.

\subsection{NIST Propagator}

The NIST National Software Reference Library, or NSRL for short, is a hash database containing all the signatures of traceable software applications. We use this database to figure out if we have files in our file system that are known to be safe, so we don't have to analyze them. We use of a tool that belongs to The Sleuth Kit \cite{tsk} called \texttt{hfind}, that looks for a given hash in the database. If hash is found, we remove the \INODE belonging to that hash from the domain.

\section{Data Persistence}
\label{database}

As mentioned before, we decided to use a database to persist parsed data. This database is created at the time of parsing the data. If the data has not been parsed before, a new database is created for the data being parsed. The database contains five things: 1) the file id (the \INODE number); 2) the file name; 3) the file path; 4) the file type and 5) the file date.

\section{Caching}
\label{caching}

This caching system was created to lessen the time it takes for the system to solve all the constraint problems it was commissioned to solve by searching in a data structure that contains all the results of previous constraint problems, including the ones that have no results. Every time the system is initiated, it checks if what we're trying to solve has been solved before, and if so, it skips the whole constraint process and just gives us the results we want, if not it runs like normal and saves the results at the end. This applies to all combination of constraints.