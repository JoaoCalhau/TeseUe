%!TEX program = xelatex
%
% ================================================================
%	Tipo de dissertação:
%		escolher entre "doutoramento" ou "mestrado"
%
%	Área científica:
%		escolher entre
%			- "ct" (ciências e tecnologia, final); "ctR" (ciências e tecnologia, rascunho);
%			- "csh" (ciências sociais e humanas, final); "cshR" (ciências sociais e humanas, rascunho);
%			- "artes" (artes, final); "artesR" (artes, rascunhos)
%
% ================================================================
%
\documentclass[mestrado,ct,12pt]{teseue}
%
%
% ================================================================
%	DOCUMENTO:
%		 
%		Língua, Título, Nome do Candidato, Curso, etc
%		Estrutura
% ================================================================
%
% ----------------------------------------------------------------
%
%	LÍNGUA DA TESE
%
%	Opções atuais:
%	- PT: Português (novo acordo ortográfico)
%	- EN: Inglês
%
\tueLINGUA{EN}
%
% ----------------------------------------------------------------
%
%	TÍTULO DA TESE
%
%	Em Português e Inglês.
%
\tueTITULO
{Digital Forensics Research Using Constraint Programming}
{Pesquisa em Forense Digital Utilizando Programação por Restrições}
%
% ----------------------------------------------------------------
%
%	SUBTÍTULO DA TESE
%
%	Em Português e Inglês.
%
\tueSUBTITULO
{}
{}
%
% ----------------------------------------------------------------
%
%	CANDIDATO
%
%	Nome completo.
%		
\tueCANDIDATO
{João Calhau}
%
% ----------------------------------------------------------------
%
%	TÍTULO E NOME DO/A ORIENTADOR/A
%
%	Designação oficial e nome do orientador/a.
%	Em geral, "Orientador" ou "Orientadora".
%
\tueORIENTADOR
{Orientador}
{Pedro Salgueiro}
%
% ----------------------------------------------------------------
%
%	SEGUNDO ORIENTADOR/A (se aplicável)
%
%	Designação oficial e nome do segundo orientador/a.
%	Em geral, "Co-orientador" ou "Co-orientadora".
%
\tueSEGUNDOORIENTADOR
{Orientador}
{Salvador Abreu}
%
% ----------------------------------------------------------------
%
%	TERCEIRO ORIENTADOR/A (se aplicável)
%
%	Designação oficial e nome do terceiro orientador/a.
%	Em geral, "Co-orientador" ou "Co-orientadora".
%
\tueTERCEIROORIENTADOR
{Orientador}
{Nuno Goes}
%
% ----------------------------------------------------------------
%
%	CURSO
%
%	Nome do curso em que se enquadra esta tese.
%
\tueCURSO
{Engenharia Informática}
%
% ----------------------------------------------------------------
%
%	ESPECIALIDADE (se aplicável)
%
%	Nome da especialidade em que se enquadra esta tese.
%
%\tueESPECIALIDADE
%{}
%
% ----------------------------------------------------------------
%
%	DEPARTAMENTO
%
%	Departamento anfitrião do curso.
%
\tueDEPARTAMENTO
{Departamento de Informática}
%
% ----------------------------------------------------------------
%
%	ESCOLA
%
%	Escola a que pertence o departamento.
%
\tueESCOLA
{Escola de Ciências e Tecnologia}
%
% ----------------------------------------------------------------
%
%	PALAVRAS CHAVE
%
%	Palavras chave da tese.
%
\tuePALAVRASCHAVE
{Digital Forensics, Constraint Programming, Security, Declarative Programming}
{Forense Digital, Programação por Restrições, Segurança, Programação Declarativa}
%
% ----------------------------------------------------------------
%
%	DATA
%
%	Data de submissão da tese.
%
\tueDATA
{\today}
%
% ----------------------------------------------------------------
%
%	DEDICATÓRIA
%
\tueDEDICATORIA
{I dedicate this to my Parents and to Flávia}
%
% ----------------------------------------------------------------
%
%	PREAMBULO
%
%	Comandos e definições para o LaTeX que devem estar **antes**
%	do texto do documento.
%
\tuePREAMBULOLATEX{
	\usepackage[figureright]{rotating}
	\usepackage{graphicx}
    \usepackage{algorithm}
    \usepackage{algpseudocode}
    \def\INODE{Inode }
    \def\INODES{Inodes }
}
%
% ----------------------------------------------------------------
%
%	PREAMBULO
%
%	Texto até à página 1. 
%
%	Por omissão os conteúdos estão definidos nos ficheiros
%		- prefacio.tex
%		- agradecimentos.tex
%		- acronimos.tex
%		- sumario.tex
%		- abstract.tex
%
%\tuePREAMBULO{}

%
% ----------------------------------------------------------------
%
%	CONTEÚDO
%
%	Texto principal da tese.
%
\tueCONTEUDO  % A partir da página 1
{
	%!TEX root = main.tex
\chapter{Introduction and Motivation}

\section{Introduction}

Digital forensics is a complex task that consists in analyzing large amounts of data, that can come from the most various digital sources, with a multitude of different tools.

In this work we present a system that makes use of the constraint programming paradigm and methods to solve digital forensics problems. Through the use of constraint programming, we are able to describe a digital forensics problem in a declarative and expressive way and efficiently find a solution to such problem, if it exists: a set of files or other resources that match the description of the digital forensic problem.

The main goal of the system presented in paper is to allow for an easy and efficient method to search for relevant information in the contents of digital equipment.

\section{Constraint Programming}

Constraints can be found in our day to day experiences, almost ubiquitous, representing the conditions that restrict our freedom of decision. Constraint programming is a powerful paradigm mostly used to solve combinatorial problems. We can consider it as a simple way to model real world problems but it can actually turn into a complex challenge when we want to find solutions for the problem that is being solved~\cite{Rossi2006}. 

\section{Digital Forensics}

Digital forensics is a very important discipline in criminal investigations, where relevant evidences are stored in digital devices. The digital forensics tools have become a vital tool to ensure we can rebuild information after a cyberattack or even if we just want to analyze any type of digital equipment~\cite{Garfinkel2010}.

\section{Digital Forensics and Constraint Programming}

\section{Experimental Evaluation}
	%!TEX root = main.tex
\chapter{Constraint Programming}

\section{Introduction}

Constraint Satisfaction, in it's most basic form, consists in finding a value for each one of a set of problems where constraints specify the restrictions related to those problems.

Constraint Satisfaction Problems have been tackled by the most various methods, from automata theory to ant algorithms and are a topic of interest in many fields of computer science~\cite{Rossi2006}.

\section{History of Constraint Programming}

Before understanding what constraint programming is we need to understand that Constraint Satisfaction is the process of finding a solution to a set of constraints that impose conditions that the variables must satisfy~\cite{Krzysztof2003}. A solution is found when a set of variables satisfies all constrains. These constraints and variables will be explained in deeper detail later in section~\ref{working}.

Constraint Satisfaction originally appeared in the field of artificial intelligence in the early 1970s. During the 1980s and 1990s constraints started being embedded into programming languages, as they were being developed, and the term constraint programing started taking form. Examples of these programming languages are \textit{Prolog} and \textit{C++}~\cite{Krzysztof2003}.

In artificial intelligence interest in constraint satisfaction developed into two streams, the language stream and the algorithm stream. The first stream refers to the side of Constraint Satisfaction that deals with algebraic equations, constraint statements and declarative languages, while the second stream refers to actual algorithms used to solve said constrains~\cite{Rossi2006}.

\section{Working with Constraint Programming}
\label{working}

There are several definitions we have to delve into to better understand how constraint programming works. We need to fully understand what a domain and a variable is and we also need to know what a constraint is.

\subsection{Variable}
\label{variable}

A variable has several known definitions, the one we want is in the field of mathematics. According to mathematics a variable is a symbol, usually an alphabetic character, and it represents a number, known as the value of the variable, which can be arbitrary, not fully specified or unknown~\cite{Menger1954}.

\subsection{Domain}

A domain, like the variable, also has various known definitions, however the one that interests us here is in the field of mathematical analysis. 

According to Hans Hahn, an open set is connected if it cannot be expressed as the sum of two open sets. An open connected set is called a domain~\cite{Hahn1921} 

In our case, a domain is an open connected set of variables.

\section{Approaches to Constraint Programming}

In constraint programming, we usually have a set of variables, which takes values from an initial domain, to which constraints are applied in order to reduce its domain, and thus reach a solution. Once a constraint is placed on the system it cannot violate another constraint previously applied. This way, we can express the requirements of the possible values of the variables~\cite{Pearson1997}.

Constraint satisfaction problems are typically solved with the help of solvers. These solvers are essentially search algorithms, usually based on backtracking techniques\cite{Knuth1997}, constraint propagation \cite{Lecoutre2010} or local search \cite{Dechter2003}. 

\subsection{Backtracking}

Backtracking is a search method that incrementally finds possible candidates to solve the problem. At the same time it removes the candidates that can not be used as a valid solution to the problem~\cite{Knuth1997}. One of the most used examples for this type of search method is the n-queens puzzle, where a set of $n$ queens should be organized, in a $n \times n$ chess board, in such a way that none of the queens can attack each other. Any partial solution that contains two queens that can attack each other is abandoned immediately.

\subsection{Constraint Propagation}

Constraint propagation starts by reducing the variable's domain, strengthening or creating new constraints, reducing the search space, leading to a problem that is easier to solve. Since this algorithm only reduces the search space of the problem variables. After completion, there is still the need to use another algorithm to solve the problem, which is now converted into a simpler problem by the propagators \cite{Lecoutre2010}. 

\subsection{Local Search}

Local search is an incomplete search method to find solutions for a problem. It consists in, iteratively, and with the help of previously defined heuristics, assigning values to the problem variables until all the constraints are satisfied. At each step of the iteration, the values of the variables are updated to values \emph{near} the previous value. The algorithm also knows the cost associated with the assignment of specific values to variables, allowing to check if a candidate solution has with a pre-defined cost~\cite{Dechter2003}.

\section{Known Tools and Other Approaches}

\subsection{Choco}

Choco is an open source and free access library dedicated to constraint programming. It is written in Java and supports several types of variables, including Integers, Booleans, Sets and Reals. It also supports several types of constraints such as AllDifferent and Count, configurable search algorithms and conflict explaining. The first version of Choco was developed in the early 2000s. A few years later, Choco 2 was developed and declared a success in the academic and industrial world. Since then, Choco has been completely re-written and in 2012 the third version of Choco was launched. The current version comes with a simpler API and is denominated Choco 4~\cite{chocoSolver}.

\subsection{Gecode}

Gecode is a free access, open source, portable, accessible and efficient programming environment used to develop systems and applications based on restrictions. Gecode, much like Choco, supports various types of variables and restrictions, among them are Integers, Float and Sets. These variables are used to model problems that are then solved with the help of constraint propagators and search algorithms \cite{MPG:M:5.1.0}~\cite{gecode}.

\subsection{Google OR-Tools}

Although Choco and Gecode are two of the most widely used libraries, there are also other new tools, such as the Google OR-Tools~\cite{ORTools}. Google Optimization Tools or OR-Tools is an interface that puts together several linear programming solver and that counts on the use of several types of algorithms such as search algorithms and graph algorithms. What this library has that is so noteworthy is the fact that it doesn't let itself be bound by one language. Although implemented in C++, it can be used in other languages like Python, C\# or Java.
	%!TEX root = main.tex
\chapter{Constraint Programming}

In this chapter we introduce the concepts of Constraint Programming, Constraint Satisfaction and \acf{CSP}. We also make a brief description of Constraint Programming history, known approaches and tools.

\section{Introduction}

This chapter introduces Constraint Programming, an alternative approach to programming which relies on a combination of techniques that deal with reasoning and computing~\cite{Apt2003}. Its definition actually comes from the field of Artificial Intelligence, and more specifically, arises from the concept of Constraint Satisfaction.

Constraint Satisfaction, in its most basic form, consists in finding a valid result for each set of problems. These sets of problems are modelled as \acp{CSP} which, in turn, specify sets of constraining relations between these problems. \acp{CSP} are defined in detail in Section~\ref{deeper}. 

\acp{CSP} have been tackled by the most various methods, from automata theory to ant algorithms and are a topic of interest in many fields of computer science~\cite{rossi2006handbook}.

\section{History of Constraint Programming}

Before understanding what Constraint Programming is, we need to understand that Constraint Satisfaction is the process of finding a solution to a set of constraints that impose conditions that the variables must satisfy~\cite{Apt2003}. A solution is found when a set of variables satisfies all constrains. These constraints and variables are described in detail in Section~\ref{deeper}.

Constraint Satisfaction originally appeared in the field of artificial intelligence in the early 1970s. During the 1980s and 1990s constraints started being embedded in programming languages, as they were being developed, and the term Constraint Programming started to take form. Examples of these programming languages are \textit{Prolog} and \textit{C++}~\cite{Apt2003}.

In the AI community interest in Constraint Satisfaction developed into two streams, the language stream and the algorithm stream. The first stream refers to the side of Constraint Satisfaction that deals with algebraic equations, constraint statements and declarative languages, while the second stream refers to the actual algorithms used to solve said constrains~\cite{rossi2006handbook}.

\section{Constraint Programming Concepts}
\label{deeper}

In this section we are going to introduce various important definitions related to \acp{CSP} as well how to model them and solve them.

\subsection{\acf{CSP}}

To solve a \ac{CSP} one must first model it. A \ac{CSP} is defined as a given set of $n$ variables to which each is assigned a well defined domain and a set of constraining relations and, in general, more than one representation of a problem as a \ac{CSP} exists. To solve a \ac{CSP} one must find all possible n-tuples, such that each $n$-tuple is an instantiation of the $n$ variables satisfying the relations~\cite{freuder1978synthesizing}. 

There are several definitions we have to delve into to better understand how \acp{CSP} work. We need to fully understand what a domain, a variable and a constraint are in this scope as well as how a \ac{CSP} is defined, modeled and solved. 

\subsection{Variable}

According to the field of elementary mathematics a variable is a symbol, usually an alphabetic character, and it represents a number, known as the value of the variable, which can be arbitrary, not fully specified or unknown~\cite{Menger1954}.

In more advanced mathematics, a variable is a symbol that denotes a mathematical object, which could be a number, a vector, a matrix or even a function~\cite{quine1960variables}. Similarly, in computer science a variable is a name (commonly an alphabetic character or a word) representing some value represented in computer memory.

\subsection{Domain}

According to Hans Hahn, from the field of mathematical analysis, an open set is connected if it cannot be expressed as the sum of two open sets. An open connected set is called a domain~\cite{hahn1921theorie}. In constraint programming, a domain is an open connected set of variables.

A bounded domain is a domain which is a bounded set, while an external domain is the interior of the complement of a bounded domain~\cite{miranda2012partial}. Carlo Miranda often used the term \textit{region} to identify an open connected set, while reserving the term \textit{domain} to identify an internally connected one.

\subsection{Constraint}

A constraint is a relation on the domains of a set of variables. It can be viewed as a requirement that states which combinations of values from the variable domains are allowed~\cite{Apt2003}.

During the modeling of a \ac{CSP} there can be two types of constraints. If the problem mandates that the constraints be satisfied, then the constraints are referred to has \textit{hard constraints}. However if in the problem it is preferred, but not required, that certain constraints be satisfied, then such non-mandatory constraints are referred to as \textit{soft constraints}.

\subsection{Modeling a \ac{CSP}}

\floatname{algorithm}{Listing}

\begin{algorithm}
    \caption{Modelling a \ac{CSP}}
    \label{modellingCSP}
    \begin{algorithmic}
        \State{}
        \State{$CSP = (V, D, C)$}
        \State{}
        \State{$V =\{V_1,V_2,\ldots,V_n\}$}
        \State{}
        \State{$D = \{D_1,D_2,\ldots,D_n\},\quad V_i \in D_i$}
        \State{}
        \State{$C = (C_1,C_2,\ldots,C_t), \quad C_j = (R_i,S_j)$}
        \State{}
    \end{algorithmic}
\end{algorithm}

Listing~\ref{modellingCSP} represents a classically modelled \ac{CSP}, according to Eugene C.Freuder and Alan K.Mackworth, as a triple $CSP = (V, D, C)$ where $V$ is an $n$-tuple of variables $V=(V_1,V_2,\ldots,V_n)$, D is a corresponding $n$-tuple of domains $D=(D_1,D_2,\ldots,D_n)$ such that $V_i \in D_i$, $C$ is a $t$-tuple of constraints $C=(C_1,C_2,\ldots,C_t)$. A constraint $C_j$ is a pair $(R_i,S_j)$ where $R_i$ is a subset of the Cartesian product of the domains of the variables in $S_i$~\cite{rossi2006handbook}.

\section{Approaches to Constraint Programming}

In Constraint Programming, we usually have a set of variables, which takes values from an initial domain, to which constraints are applied in order to reduce its domain, and thus reach a solution. Once a constraint is placed on the system it cannot violate another constraint previously applied. This way, we can express the requirements of the possible values of the variables~\cite{Pearson1997}.

\ac{CSP} are typically solved with the help of solvers. These solvers are essentially search algorithms, usually based on backtracking techniques\cite{Knuth1997}, constraint propagation \cite{Lecoutre2010} or local search \cite{Dechter2003}. 

\subsection{Backtracking}

Backtracking search is a basic control structure in computer science and finds a particular application in artificial intelligence. As the search is in general exponential in the number of variables, it is particularly helpful to have some means of obtaining bounds on the effort required for individual problems or classes of problems~\cite{freuder1985sufficient}. 

Backtracking searches incrementally and finds possible candidates to solve the problem. At the same time, it removes the candidates that can not be used as a valid solution to the problem~\cite{Knuth1997}. One of the most used examples for this type of search method is the n-queens puzzle, where a set of $n$ queens should be organized, in a $n \times n$ chess board, in such a way that none of the queens can attack each other. Any partial solution that contains two queens that can attack each other is abandoned immediately. The modelling can be seen in Section~\ref{nqueens}.

\subsection{Constraint Propagation}

Constraint propagation is the basic operation in Constraint Programming. It is well-recognized that its extensive use is necessary when efficiently solving hard \acp{CSP}. Almost all constraint solvers use it as a basic step~\cite{bessiere2001refining}.

Constraint propagation uses semantics of constraints to identify and discard incompatible combinations of values. Such combinations, which are instantiations that cannot be part of any solution, are called \textit{nogoods}. A typical way to propagate constraints is by reducing the domains of variables or the relations of constraints, resulting in a simpler search space to work on because all the \textit{nogoods} are now identified and can be used to avoid exploration of useless parts in the search space. 

Constraint propagation cannot, however, be used on it's own because it is only used to simplify a problem while maintaining its semantics so as to make it easier to solve. Some other algorithm must be then used to find the solutions or even optimal solutions~\cite{Lecoutre2010}.

\subsection{Local Search}

Local search methods date back over thirty years ago. Applied to difficult combinatorial optimization problems, this heuristic approach yields high-quality solutions by iteratively considering small modifications of a good solution in the hope of finding a better one~\cite{lin1965computer}.

Local search methods generally involve going from one solution to another repeatedly, achievable only through a local move, where valid moves can vary from problem to problem. The set of all solutions, reachable from the initial solution $S$ through a local move is called the neighborhood of $S$. The set of all probable solutions in the neighborhood of $S$ is called its probable neighborhood. To find the optimal solution, a strategy called \textit{iterative improvement} moves, on each iteration, to the best probable neighbor (the least costly) until it can't improve on the current solution~\cite{jussien2002local}.

Of course, this strategy has a very obvious drawback, and because of this several ways of alleviating the drawback have been proposed, such as multi-start iterative improvements and genetic local search~\cite{holland1992adaptation}. Both of these apply iterative improvements. The first one builds a pool of solutions and returns the best one while the second one discards the least-promising solutions and repeats the process until some stopping criteria is satisfied~\cite{jussien2002local}.

\section{Known Constraint Programming Solvers}

To reach a solution to a \ac{CSP} a Solver is needed. A various number of Constraint Programming Solvers exist, we present some of the most used ones in this section. Special attention should be placed on Choco, because this is the tool we chose to use in our constraint solving problem.

\subsection{Choco}

Choco is an open source and free access library dedicated to Constraint Programming. It is written in Java and supports several types of variables, including Integers, Booleans, Sets and Reals. It also supports several types of constraints such as AllDifferent and Count, configurable search algorithms and conflict explaining. It aims at describing real combinatorial problems in the form of \acp{CSP} and solving them with Constraint Programming techniques. Choco is used for teaching, researching and real life-applications~\cite{aboutChoco}~\cite{chocoSolver}.

The first version of Choco was developed in the early 2000s. A few years later, Choco 2 was developed and declared a success in the academic and industrial world. Since then, Choco has been completely re-written and in 2012 the third version of Choco was launched. The current version comes with a simpler API and is called Choco~\cite{historyChoco}~\cite{chocoSolver}.

Choco is among the fastest \ac{CSP} solvers on the market. In 2013 and 2014, Choco was awarded two silver medals and three bronze medals at the MiniZinc Challenge that is the world-wide competition of Constraint Programming solvers~\cite{aboutChoco}~\cite{chocoSolver}.

\subsection{Gecode}

Gecode is a free access, open source, portable, accessible and efficient programming environment used to develop systems and applications based on Constraints. Gecode, much like Choco, supports various types of variables and constraints, among them are Integers, Float and Sets. These variables are used to model problems that are then solved with the help of constraint propagators and search algorithms~\cite{schulte2010modeling}~\cite{gecode}.

Gecode is one of the fastest \ac{CSP} solvers, it won all gold medals in all categories in the MiniZinc Challenges from 2008 to 2012~\cite{gecode}.

\subsection{Google OR-Tools}

Although Choco and Gecode are two of the most widely used libraries, there are also other new tools, such as the Google OR-Tools~\cite{ORTools}. Google Optimization Tools or OR-Tools is an interface that puts together several linear programming solver and that counts on the use of several types of algorithms such as search algorithms and graph algorithms. What this library has that is so noteworthy is the fact that it doesn't let itself be bound by one language. Although implemented in C++, it can be used in other languages like Python, C\# or Java~\cite{ORTools}.

\section{Classic Constraint Problems}

In this section we present some classic constraint problems along with their modelling.

\subsection{$N$ Queens}
\label{nqueens}

The $n$ queens problem is, fundamentally, placing $n$ chess queens on an $n \times n$ chessboard so that no two queens attack each other~\cite{rivin1994n}. A feasible solution for this problem would be one in which no two queens share the same row, column or diagonal~\cite{hoffman1969constructions}.

The original $n$ queens problem was introduced by Gauss, in 1850, with an $n$ value of 8~\cite{hoffman1969constructions} and a total of 72 solutions, but soon after 92 solutions where theorized and in 1874 proved~\cite{rivin1994n}. For a board like this, one of the possible solutions can be seen in figure~\ref{fig:queens}.

\begin{figure}[hb]
    \centering
    \includegraphics[width=50mm]{queens.png}
    \caption{Possible solution to the 8 queens problem}
    \label{fig:queens}
\end{figure}

By treating the chessboard as an $n \times n$ matrix of square elements, any square can be identified by an ordered pair $(i, j)$ where $i$ and $j$ are row and column numbers respectively. A major diagonal of the matrix can be identified by $m-j+i = CONSTANT$ where the $CONSTANT$ is the number of the diagonal. We can further define a minor diagonal with $i+j-1 = CONSTANT$ where $CONSTANT$ is the number of the diagonal~\cite{hoffman1969constructions}. 
The following constraints must always be applied to the $n$ queens problem:
\begin{itemize}
    \item a) The row numbers are unique.
    \item b) The column numbers are unique.
    \item c) The major diagonal numbers are unique.
    \item d) The minor diagonal numbers are unique.
\end{itemize}

\subsection{Sudoku}

According to J. Scott Provan, Sudoku is a logic-based, combinatorial number placement puzzle that can be defined by a set $S$ of $n$ grid squares that each have a possible placement of numbers from a set $P=\{1,\ldots,m\}$, a collection of blocks $\beta$ where each block consists of a set of exactly $m$ squares and an initial assignment of numbers in some grid squares. The goal is to assign numbers from $P$ to each of the remaining squares of $S$ in a way that each block $\beta$ has a complete set $P$ of non repeated numbers~\cite{provan2009sudoku}.

The typical Sudoku puzzle has a $m \times m$ configuration, and the usual value of $m$ is 9, which brings the value of $n$ to a total of 81 grid squares. On top of having to comply with the constraint of non repeating numbers in a block, the typical Sudoku puzzle also can not have repeating numbers in each row and column.

One requirement is that every Sudoku puzzle must have exactly one solution~\cite{provan2009sudoku}. An example of a standard Sudoku puzzle can be seen in figure~\ref{fig:sudoku} along with it's solution.

\begin{figure}[hb]
    \centering
    \includegraphics[width=120mm]{sudoku.png}
    \caption{Sudoku puzzle with respective solution}
    \label{fig:sudoku}
\end{figure}
	%!TEX root = main.tex
\chapter{Our Approach}

\section{Introduction}

This section introduces our approach for modeling a Digital Forensics Problem as a Constraint Satisfaction Problem and reaching a valid solution using the Choco Solver. We describe how we modelled the problem as a CSP, the methodologies used to analyze and extract the information from the digital evidences, the data structures used to model the problem and how they are used to reach a solution.

\section{Methodology}

After acquiring the disk image to be analyzed, it is first processed with the help of tools from The Sleuth Kit~\cite{tsk}, first with Sorter and then with Mactime. Sorter creates multiple files with different names, each name being a pre-determined type of file, such as archive, executable or data. As for Mactime, it creates a single file containing time information about every file present in the file system. Examples of these files can be seen in Figure~\ref{fig:sorterOut} and Figure~\ref{fig:mactimeOut}. 

The files created by Sorter include all relevant information about the files present in the file system that is being analyzed, including: file path in the file system, the file type, the image name (from where the data was extracted) and the \INODE number, which is an internal representation of that particular file in the file system. As for the Mactime file, it only contains date information about each file. This information is parsed into a database that allows the data to be persistent. This database is described in detail in Section~\ref{database}. The data also passes through a caching system that tries to determine if the exact type of constraints have been applied to the file system being analyzed to find if it can skip the lengthily process of trying to find a solution. The caching system is described in detail in Section~\ref{caching}.

\begin{figure}
    \centering
    \includegraphics[width=120mm]{sorter_out.png}
    \caption{Example of Sorter output for executable files}
    \label{fig:sorterOut}
\end{figure}

\begin{figure}
    \centering
    \includegraphics[width=120mm]{mactimeOut.png}
    \caption{Example of Mactime output for a 4GB pen drive}
    \label{fig:mactimeOut}
\end{figure}

The whole program can be shortened into the small flow diagram seen in figure \ref{fig:diagram}.

\begin{figure}
    \centering
    \includegraphics[width=120mm]{diagram.png}
    \caption{Flow diagram}
    \label{fig:diagram}
\end{figure}

After pre-processing the disk image and persisting all necessary data, the digital forensics problem, which describes the items that being looked for, is modeled as a Constraint Satisfaction Problem (CSP). This CSP is then solved by a CSP solver, to reach a solution that satisfies the initial digital forensics problem, if it exists. A solution to such a CSP will be a set of file identifiers that match the digital forensics problem. It is necessary to use specific constraints to reach the solution mentioned before, which are used to restrict the domain of the variables. These constraints are described in Section~\ref{constraints}.

\section{Modeling}

The problem is modeled as a set of variables that represent the files that need to be found, according to the digital forensics problem. These files are represented as numerical values, which are the \INODE numbers extracted from the Sorter mentioned previously. Each of these variables are associated with a domain and a set of constraints which are applied to the variables. As previously stated these constraints were created specifically in the context of this work.

In this work we use the Choco Solver, which allows the use of four different types of variables to model the problems: Integers (IntVar), Booleans (BoolVar), Sets (SetVar) and Reals (RealVar). To model a digital forensics problem as a CSP, we decided to use  Set variables (SetVars).

In Choco Solver, SetVars are defined by a domain that is made of two separate domains, the Lower Bound (LB) and the Upper Bound (UB). The Lower Bound is a set of integers that must belong to every solution, while the Upper Bound is composed of the set of integers that may be part of the final solution \cite{SetVar}. In our case, when creating the variable with which we are going to work with, the Lower Bound will be left empty, because we do not know what files we want yet. As for the  Upper Bound, it will be composed of all the \INODES found during the parsing phase.

\floatname{algorithm}{Listing}

\begin{algorithm}
    \caption{Modelling of a Digital Forensics CSP}
    \label{modelling}
    \begin{algorithmic}
        \State{}
        \State{$CSP = (V, D, C)$}
        \State{}
        \State{$V =\{V_1,V_2,\ldots,V_n\}$}
        \State{}
        \State{$D = \{D_1,D_2,\ldots,D_n\},\quad \forall D_i \in D : D_i = \{Y_1,Y_2,\ldots,Y_z\}$}
        \State{}
        \State{$C=\{C_1(V_i,\ldots,V_j),\ldots,C_k(V_i,\ldots,V_j)\},$}
        \State{$\qquad \qquad \qquad \qquad \qquad \forall C_k \in C : C_k = \{CP_1(V_i,\ldots,V_j),\ldots,CP_z(V_i,\ldots,V_j)\}$}
        \State{}
    \end{algorithmic}
\end{algorithm}

Listing~\ref{modelling} presents a formal representation of a digital forensics CSP, represented by the triple $CSP = (V,D,C)$. $V$ is the set of variables, which represents the files to be found, $D$ is the set of domains for each variable which, where each $D_i = \{Y_1,Y_2,\ldots,Y_z\}$ is a set of integer values that represent the \INODES present in the file system; and $C$ is the set of constraints which restricts the domain of each variable, where each $C_k$ is mapped into a specific constraint propagator.

\section{Constraints}
\label{Constraints}

To reach a solution, Choco solver makes use of constraints which are implemented as propagators. A propagator declares a filtering algorithm that can be applied to the variables that models the problem, in order to reduce their domain~\cite{Propagator}. In the context of this work, we implemented the following propagators:

\begin{enumerate}
    \item \textbf{File type:} restricts our domain according to the given type of file passed as argument.
    \item \textbf{File path:} restricts our domain according to the given path passed as argument.
    \item \textbf{Word search:} restricts our domain according to the given word passed as argument.
    \item \textbf{NIST:} restricts our domain according to the NIST database.
    \item \textbf{Date:} restricts our domain according to the given date passed as argument.
\end{enumerate}

All these propagators were implemented to work with the variables used to model the problem: SetVars.

To create the propagators, we had to implement the following methods: \texttt{propagate} and \texttt{isEntailed}. The method \texttt{propagate} should restrict the domain according to what we need, while the method \texttt{isEntailed} informs the propagator if a problem has a solution or not, or if it is not possible to determine if a solution exists. These methods are described in detail in Listings \ref{propagate} and \ref{isEntailed}.

\floatname{algorithm}{Listing}

\begin{algorithm}
    \caption{propagate method}
    \label{propagate}
    \begin{algorithmic}
        \For{each value in UB}
            \If{value does not belong to desired structure}
                \State{Remove value from UB}
            \EndIf
        \EndFor
    \end{algorithmic}
\end{algorithm}

\begin{algorithm}
    \caption{isEntailed method}
    \label{isEntailed}
    \begin{algorithmic}
        \If{UB is empty}
            \State{Problem is impossible to solve}
        \Else
            \State{Problem has possible solution}
        \EndIf
    \end{algorithmic}
\end{algorithm}

\subsection{Type, Path and Date Propagators}

The Type propagator was the first propagator created and it takes the Upper Bound of the SetVar, iterates over it and removes any \INODE that does not belong to the type we are looking for. Path and Date propagators work in the same way. The main difference is what they are trying to restrict: the Type propagator receives the type of file we want to restrict and iterates over the Upper Bound to remove every \INODE that does not belong to that type of file, while the Path propagator receives a path and, similarly to the Type propagator, iterates over the Upper Bound and removes every \INODE that does not belong to that path.

\subsection{Word Search Propagator}

This propagator relies on the Unix4j~\cite{Unix4j} Java library. This library implements most of the Linux bash commands in native Java. We use this library since it provides efficient methods to find contents inside a file in any file system.

This propagator works in a similar way to the ones previously described. It iterates over the Upper Bound and removes any file that does not contain the word we passed as argument. Also, since this is the last constraint ever being run, we add the \INODES that passed through the propagator to the Lower Bound, because at this moment we're certain these will belong to the final solution.

\subsection{NIST Propagator}

The NIST National Software Reference Library, or NSRL for short, is a hash database containing all the signatures of traceable software applications. We use this database to figure out if we have files in our file system that are known to be safe, so we don't have to analyze them. We use of a tool that belongs to The Sleuth Kit \cite{tsk} called \texttt{hfind}, that looks for a given hash in the database. If hash is found, we remove the \INODE belonging to that hash from the domain.

\section{Data Persistence}
\label{database}

As mentioned before, we decided to use a database to persist parsed data. This database is created at the time of parsing the data. If the data has not been parsed before, a new database is created for the data being parsed. The database contains five things: 1) the file id (the \INODE number); 2) the file name; 3) the file path; 4) the file type and 5) the file date.

\section{Caching}
\label{caching}

This caching system was created to lessen the time it takes for the system to solve all the constraint problems it was commissioned to solve by searching in a data structure that contains all the results of previous constraint problems, including the ones that have no results. Every time the system is initiated, it checks if what we're trying to solve has been solved before, and if so, it skips the whole constraint process and just gives us the results we want, if not it runs like normal and saves the results at the end. This applies to all combination of constraints.
	%!TEX root = main.tex
\chapter{Experimental Evaluation}

In this chapter we are going to talk about the results of the experimental evaluation of the initial program phase. At the same time, we will also present the results of the experimental evaluation of the current development phase.

\section{Introduction}

This chapter presents the experimental results that resulted from two distinct program phases, the initial phase and the current phase. Along with the experimental results, we also present the method through which we extracted the disk images used in said experiments along with a few notes depicting what we could conclude from those results.

\section{Initial Experimental Results}

To evaluate the initial system, we used device images from personal items, including a USB pen drive and an SD card that came from a smartphone. To safely extract and analyze the images from these devices, we used \ac{FTK}. After acquiring the images, they were processed with the \ac{TSK} toolkit sorter to extract their information. They were then sorted into their respective structures and analyzed with the tool developed in the context of this work. The USB pen drive and SD card images have a 4GB and 32GB total size, respectively.

\begin{table}[ht]
    \centering
    \begin{tabular}{|p{3cm}||p{1cm}|p{2cm}|p{1.5cm}|p{3cm}|}
        \hline
        Device Name & Time (sec.) & Total \INODES & Found \INODES & Constraints \\
        \hline
        Pen Drive & 0.343 & 113 & 2 & type \\
        Pen Drive & 0.345 & 113 & 6 & path \\
        Pen Drive & 0.387 & 113 & 1 & type, path \\
        \hline
        SD Card  & 36.081 & 3060 & 1758 & type \\
        SD Card  & 28.364 & 3060 & 6 & path \\
        SD Card  & 24.905 & 3060 & 6 & type, path \\
        \hline
    \end{tabular}
    \caption{Initial Experimental Results}
    \label{tab:old_results}
\end{table}

The initial evaluation results are presented in table~\ref{tab:old_results}, which describes the elapsed time to reach a solution in seconds, the total number of files present in the device, number of \INODES Found that match the modeled \ac{DFP} and the constraints used to model the problem. 

We can note that, in this version of the tool, time scaled really poorly with the amount of \INODES to process. This is due to the fact that each time we want to process the \INODES with the tool, they need to be passed primarily through each data structure, which are only located in-memory, and then processed with Choco Solver. This process is a lengthy one, even more so with an increase in \INODE quantities, a number that in a real world scenario could be in the thousands.

Note that in this early version of the program there were no date and word search propagators, so they are not present in this test. Also no caching system existed, that is why it is not mentioned either.


The constraints used for the pen drive, from top to bottom, are as follow:
\begin{itemize}
    \item 1st test: File type was "Unknown".
    \item 2nd test: File path was "LVOC/LVOC".
    \item 3rd test: File type was "Unknown" and file path was "LVOC/LVOC".
\end{itemize}

The constraints used for the SD card, from top to bottom, are as follow:
\begin{itemize}
    \item 1st test: File type was "Audio".
    \item 2nd test: File path was "Music/Disturbed".
    \item 3rd test: File type was "Audio" and file path was "Music/Disturbed". 
\end{itemize}

\section{Current Experimental Results}

To evaluate the final version of the tool developed in the context of this work, we used device images from different sources such as, Digital Corpora~\cite{DCorpora} and Linux LEO~\cite{LinuxLEO}. The images taken from Digital Corpora are two, one from a Canon camera, code named "canon2" and the other a bootable USB disk with a ext3 Ubuntu file system installed, code named "casper". From Linux LEO one image was taken, an Able2 Ext2 disk image, code named "able2".

\begin{table}[ht]
    \centering
    \begin{tabular}{|p{3cm}||p{1cm}|p{2cm}|p{1.5cm}|p{3.5cm}|p{1.5cm}|}
        \hline
        Device Name & Time (sec.) & Total \INODES & Found \INODES & Constraints & Caching \\
        \hline
        Canon2  & 0.745 & 38 & 33 & type & no  \\
        Canon2  & 0.677 & 38 & 33 & type & yes \\
        Canon2  & 1.254 & 38 & 36 & path & no  \\
        Canon2  & 0.836 & 38 & 36 & path & yes \\
        Canon2  & 0.942 & 38 & 33 & type, path & no \\
        Canon2  & 0.932 & 38 & 33 & type, path & yes \\
        Canon2  & 0.911 & 38 & 33 & type, path, date & no \\
        Canon2  & 0.667 & 38 & 33 & type, path, date & yes \\
        Canon2  & 0.983 & 38 & 33 & type, path, date, word & no \\
        Canon2  & 0.930 & 38 & 33 & type, path, date, word & yes \\
        \hline
        Casper  & 1.473 & 1079 & 79 & type & no  \\
        Casper  & 0.865 & 1079 & 79 & type & yes \\
        Casper  & 1.782 & 1079 & 5 & path & no  \\
        Casper  & 0.901 & 1079 & 5 & path & yes \\
        Casper  & 1.675 & 1079 & 4 & type, path & no \\
        Casper  & 0.819 & 1079 & 4 & type, path & yes \\
        Casper  & 1.482 & 1079 & 4 & type, path, date & no \\
        Casper  & 0.767 & 1079 & 4 & type, path, date & yes \\
        Casper  & 1.632 & 1079 & 4 & type, path, date, word & no \\
        Casper  & 0.752 & 1079 & 4 & type, path, date, word & yes \\
        \hline
        Able2 Partition 3  & 29.539 & 11653 & 14 & type & no  \\
        Able2 Partition 3  & 0.974 & 11653 & 14 & type & yes \\
        Able2 Partition 3  & 37.010 & 11653 & 66 & path & no  \\
        Able2 Partition 3  & 0.924 & 11653 & 66 & path & yes \\
        Able2 Partition 3  & 28.334 & 11653 & 12 & type, path & no \\
        Able2 Partition 3  & 0.881 & 11653 & 12 & type, path & yes \\
        Able2 Partition 3  & 26.858 & 11653 & 12 & type, path, date & no \\
        Able2 Partition 3  & 0.862 & 11653 & 12 & type, path, date & yes \\
        Able2 Partition 3  & 28.584 & 11653 & 2 & type, path, date, word & no \\
        Able2 Partition 3  & 0.885 & 11653 & 2 & type, path, date, word & yes \\
        \hline
    \end{tabular}
    \caption{Actual Experimental Results}
    \label{tab:results}
\end{table}

The actual evaluation results are presented in table~\ref{tab:results}, which describes the elapsed time to reach a solution in seconds, the total number of files present in the device, number of \INODES found that match the modeled Digital Forensics problem, the constraints used to model the problem and if the caching system was used, for each experiment.

When comparing the two experimental results we can clearly see that the current version of the tool is quite a bit faster than the first implementation, of course this is due to the fact that there is no need to store the \INODES in various structures. This implementation is both faster in terms of processing speed and data persistence, of course these times could be minimized even further if we implemented a database that made use of a faster search method, like for example elastisearch. 

The constraints used for the canon2 image, from top to bottom, are as follow:
\begin{itemize}
    \item 1st and 2nd tests: File type was "Image".
    \item 3rd and 4th tests: File path was "DCIM/100CANON".
    \item 5th and 6th tests: File type was "Image" and file path was "DCIM/100CANON".
    \item 7th and 8th tests: File type was "Image", file path was "DCIM/100CANON" and file date was "2018-01-01" and onwards.
    \item 9th and 10th tests: File type was "Image", file path was "DCIM/100CANON", file date was "2018-01-01" and onwards and word to look for was ."IMG".
\end{itemize}

The constraints used for the casper image, from top to bottom, are as follow:
\begin{itemize}
    \item 1st and 2nd tests: File type was "Documents".
    \item 3rd and 4th tests: File path was "home/ubuntu/Documents/ssa.gov".
    \item 5th and 6th tests: File type was "Documents" and file path was "home/ubuntu/Documents/ssa.gov".
    \item 7th and 8th tests: File type was "Documents", file path was "home/ubuntu/Documents/ssa.gov" and file date was "2008-12-30" and onwards.
    \item 9th and 10th tests: File type was "Documents", file path was "home/ubuntu/Documents/ssa.gov", file date was "2008-12-30" and onwards and word to look for was "pdf".
\end{itemize}

The constraints used for the able2 partition 3 image, from top to bottom, are as follow:
\begin{itemize}
    \item 1st and 2nd tests: File type was "Archive".
    \item 3rd and 4th tests: File path was "lib".
    \item 5th and 6th tests: File type was "Archive" and file path was "lib".
    \item 7th and 8th tests: File type was "Archive", file path was "lib" and file date was "2000-01-01" and onwards.
    \item 9th and 10th tests: File type was "Archive", file path was lib", file date was "2000-01-01" and onwards and word to look for was "libc".
\end{itemize}
	%!TEX root = main.tex
\chapter{Conclusion and Future Work}

In this dissertation we described a new approach to Digital Forensics, which relies on Declarative Programming approaches, more specifically Constraint Programming, to find evidences or elements in digital equipment that might be related to criminal activities. With this approach, we were able to describe several Digital Forensics problems in a descriptive and intuitive way, and, based on that description, use CSP solvers to find a set of evidences/elements that match the initial problem description in a useful time frame.

The final version of the system presented in this dissertation is quite different from the initial idea, where we had different data structures where the data was saved, no disk persistence what so ever and no caching system. However, the current version can take a correctly modelled \ac{DFP} and make use of the Constraint Programming paradigm to find its solutions in a way that is intuitive, easy to use and fairly fast to execute. Beyond this there are a couple of other assets that help make the system more reliable: the database, that persists data and the caching system that speeds up the solution finding. 
Nonetheless, there are some aspects that need to be improved. In a near future, we will use a word indexing system for the word search propagator to improve its performance. We might also consider the creation of special propagators to find hidden messages in image files, as well as the creation a \acf{DSL} to make the description of the problems easier for the Digital Forensics analysts. Other than these propositions we could also develop an improvement for the caching system to shorten the processing time even further and a tool to integrate everything.
}
%
% ----------------------------------------------------------------
%
%	APÊNDICES
%
%	Texto complementar da tese.
%
%\tueAPENDICES % Material de suporte
%{
%	\include{demo-apend-1}
%	\include{demo-apend-2}
%}
%
% ----------------------------------------------------------------
%
%	BIBLIOGRAFIA
%
%	Por omissão...
%	- usa BibTex
%	- com o estilo "alpha"
%	- consulta o ficheiro "bibliografia.tex"
%	- lista **todas** as obras, mesmo que não referenciadas no texto da tese
%
%\tueBIBLIOGRAFIA{}
%
% ----------------------------------------------------------------
%
%	ÍNDICE REMISSIVO
%
%\tueINDICEREMISSIVO{}
%
% ----------------------------------------------------------------
%
% ================================================================
%
%	Modo ORGANIZAÇÃO DA DISSETAÇÃO COMPLETA.
%
%	Prevê que
%		- a informação sobre título, autor, orientadores, etc está definida acima e que
%		- a obra tem a seguinte estrutura:
%
%			prefácio
%			agradecimentos
%			tabela de conteúdos
%			lista de figuras
%			lista de tabelas
%			lista de acrónimos
%			sumário
%			tradução do sumário
%			------------------------------
%			CONTEÚDO (vários capítulos)
%			APÊNDICES (vários capítulos)
%			------------------------------
%			bibliografia
%			índice remissivo
%
% ================================================================
%
\tueDOCUMENTO
%
% ================================================================
%	Modo CAPA, CONTRA-CAPA e LOMBADAS.
%
%	Prevê que a informação sobre título, autor, orientadores, etc está definida acima.
%
% ================================================================
%
%\tueCAPAS

