%!TEX root = main.tex
\chapter{Conclusion and Future Work}

In this dissertation we described a new approach to Digital Forensics, which relies on Declarative Programming approaches, more specifically Constraint Programming, to find evidences or elements in digital equipment that might be related to criminal activities. With this approach, we were able to describe several Digital Forensics problems in a descriptive and intuitive way, and, based on that description, use CSP solvers to find a set of evidences/elements that match the initial problem description in a useful time frame.

The final version of the system presented in this dissertation is quite different from the initial idea, where we had different data structures where the data was saved, no disk persistence what so ever and no caching system. However, the current version can take a correctly modelled \ac{DFP} and make use of the Constraint Programming paradigm to find its solutions in a way that is intuitive, easy to use and fairly fast to execute. Beyond this there are a couple of other assets that help make the system more reliable: the database, that persists data and the caching system that speeds up the solution finding. 
Nonetheless, there are some aspects that need to be improved. In a near future, we will use a word indexing system for the word search propagator to improve its performance. We might also consider the creation of special propagators to find hidden messages in image files, as well as the creation a \acf{DSL} to make the description of the problems easier for the Digital Forensics analysts. Other than these propositions we could also develop an improvement for the caching system to shorten the processing time even further and a tool to integrate everything.