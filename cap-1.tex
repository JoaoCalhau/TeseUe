%!TEX root = main.tex
\chapter{Introduction and Motivation}

\section{Introduction}

Digital forensics is a complex task that consists in analyzing large amounts of data, that can come from the most various digital sources, with a multitude of different tools.

In this work we present a system that makes use of the constraint programming paradigm and methods to solve digital forensics problems. Through the use of constraint programming, we are able to describe a digital forensics problem in a declarative and expressive way and efficiently find a solution to such problem, if it exists: a set of files or other resources that match the description of the digital forensic problem.

The main goal of the system presented in paper is to allow for an easy and efficient method to search for relevant information in the contents of digital equipment.

\section{Constraint Programming}

Constraints can be found in our day to day experiences, almost ubiquitous, representing the conditions that restrict our freedom of decision. Constraint programming is a powerful paradigm mostly used to solve combinatorial problems. We can consider it as a simple way to model real world problems but it can actually turn into a complex challenge when we want to find solutions for the problem that is being solved~\cite{Rossi2006}. 

\section{Digital Forensics}

Digital forensics is a very important discipline in criminal investigations, where relevant evidences are stored in digital devices. The digital forensics tools have become a vital tool to ensure we can rebuild information after a cyberattack or even if we just want to analyze any type of digital equipment~\cite{Garfinkel2010}.

\section{Digital Forensics and Constraint Programming}

\section{Experimental Evaluation}