%!TEX root = main.tex
\chapter{Experimental Evaluation}

\section{Introduction}

\section{Experimental Results}

To evaluate the system, we used device images from different sources that claim their images are for the testing of forensics computer tools, Digital Corpora~\cite{DCorpora} and Linux LEO~\cite{LinuxLEO}. The images taken from Digital Corpora are two, one from a Canon camera, code named "canon2" and the other a bootable USB disk with a ext3 Ubuntu file system installed, code named "casper". From Linux LEO one image was taken, an Able2 Ext2 disk image, code named "able2".

\begin{table}
    \centering
    \begin{tabular}{|p{3cm}||p{1cm}|p{1.5cm}|p{1cm}|p{3.5cm}|p{1.5cm}|}
        \hline
        Device Name & Times (sec.) & Total \INODES & Found \INODES & Constraints & Caching \\
        \hline
        Canon2  & 1.6 & 38 & 33 & type, path & no  \\
        Canon2  & 1.0 & 38 & 33 & type, path & yes \\
        Canon2  & 1.8 & 38 & 33 & type, path, date & no \\
        Canon2  & 0.6 & 38 & 33 & type, path, date & yes \\
        Canon2  & 1.4 & 38 & 33 & type, path, date, word & no \\
        Canon2  & 0.6 & 38 & 33 & type, path, date, word & yes \\
        \hline
        Casper  & 2.6 & 1079 & 4 & type, path & no  \\
        Casper  & 0.5 & 1079 & 4 & type, path & yes \\
        Casper  & 2.7 & 1079 & 4 & type, path, date & no \\
        Casper  & 0.7 & 1079 & 4 & type, path, date & yes \\
        Casper  & 2.7 & 1079 & 4 & type, path, date, word & no \\
        Casper  & 0.7 & 1079 & 4 & type, path, date, word & yes \\
        \hline
        Able2 Partition 3 & 88.2 & 11653 & 12 & type, path & no  \\
        Able2 Partition 3 & 0.7  & 11653 & 12 & type, path & yes \\
        Able2 Partition 3 & 66.4 & 11653 & 12 & type, path, date & no  \\
        Able2 Partition 3 & 1.1  & 11653 & 12 & type, path, date & yes \\
        Able2 Partition 3 & 61.8 & 11653 & 12 & type, path, date, word & no  \\
        Able2 Partition 3 & 1.1  & 11653 & 12 & type, path, date, word & yes \\
        \hline
    \end{tabular}
    \caption{Experimental Results}
    \label{tab:results}
\end{table}

The evaluation results are presented in Table~\ref{tab:results}, which describes the elapsed time to reach a solution in seconds, the total number of files present in the device, number of \INODES found that match the modeled Digital Forensics problem, the constraints used to model the problem and if the caching system was used, for each experiment.